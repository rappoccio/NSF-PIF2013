


<!DOCTYPE html>
<html>
  <head prefix="og: http://ogp.me/ns# fb: http://ogp.me/ns/fb# githubog: http://ogp.me/ns/fb/githubog#">
    <meta charset='utf-8'>
    <meta http-equiv="X-UA-Compatible" content="IE=edge">
        <title>Exascale/CollaborativeFund2013/collaborative_proposal.tex at master · rappoccio/Exascale · GitHub</title>
    <link rel="search" type="application/opensearchdescription+xml" href="/opensearch.xml" title="GitHub" />
    <link rel="fluid-icon" href="https://github.com/fluidicon.png" title="GitHub" />
    <link rel="apple-touch-icon" sizes="57x57" href="/apple-touch-icon-114.png" />
    <link rel="apple-touch-icon" sizes="114x114" href="/apple-touch-icon-114.png" />
    <link rel="apple-touch-icon" sizes="72x72" href="/apple-touch-icon-144.png" />
    <link rel="apple-touch-icon" sizes="144x144" href="/apple-touch-icon-144.png" />
    <link rel="logo" type="image/svg" href="https://github-media-downloads.s3.amazonaws.com/github-logo.svg" />
    <meta property="og:image" content="https://github.global.ssl.fastly.net/images/modules/logos_page/Octocat.png">
    <meta name="hostname" content="github-fe120-cp1-prd.iad.github.net">
    <meta name="ruby" content="ruby 1.9.3p194-tcs-github-tcmalloc (2012-05-25, TCS patched 2012-05-27, GitHub v1.0.36) [x86_64-linux]">
    <link rel="assets" href="https://github.global.ssl.fastly.net/">
    <link rel="conduit-xhr" href="https://ghconduit.com:25035/">
    <link rel="xhr-socket" href="/_sockets" />
    


    <meta name="msapplication-TileImage" content="/windows-tile.png" />
    <meta name="msapplication-TileColor" content="#ffffff" />
    <meta name="selected-link" value="repo_source" data-pjax-transient />
    <meta content="collector.githubapp.com" name="octolytics-host" /><meta content="github" name="octolytics-app-id" /><meta content="80CD110F:6828:266EC6B:52433046" name="octolytics-dimension-request_id" />
    

    
    
    <link rel="icon" type="image/x-icon" href="/favicon.ico" />

    <meta content="authenticity_token" name="csrf-param" />
<meta content="Eu+WrymALTown02sl9Yq8OURSZKgRGmSszY5c7VWYlk=" name="csrf-token" />

    <link href="https://github.global.ssl.fastly.net/assets/github-ce4aebcd33ad2703dd0722feec241e8e9460a082.css" media="all" rel="stylesheet" type="text/css" />
    <link href="https://github.global.ssl.fastly.net/assets/github2-546255f1b8702b827f9dba35e9fd3259aac96aef.css" media="all" rel="stylesheet" type="text/css" />
    

    

      <script src="https://github.global.ssl.fastly.net/assets/frameworks-4e5aeedcc7a86dcff8294cb84644a333b46202a2.js" type="text/javascript"></script>
      <script src="https://github.global.ssl.fastly.net/assets/github-eacf6a6bb5a8468ca41c9c7360a0a73c8268153e.js" type="text/javascript"></script>
      
      <meta http-equiv="x-pjax-version" content="89650f1b11a2dd056ce8ebb1135c6277">

        <link data-pjax-transient rel='permalink' href='/rappoccio/Exascale/blob/5da0c6ccb400654ab609c7b8cc42f42112c90d05/CollaborativeFund2013/collaborative_proposal.tex'>
  <meta property="og:title" content="Exascale"/>
  <meta property="og:type" content="githubog:gitrepository"/>
  <meta property="og:url" content="https://github.com/rappoccio/Exascale"/>
  <meta property="og:image" content="https://github.global.ssl.fastly.net/images/gravatars/gravatar-user-420.png"/>
  <meta property="og:site_name" content="GitHub"/>
  <meta property="og:description" content="Contribute to Exascale development by creating an account on GitHub."/>

  <meta name="description" content="Contribute to Exascale development by creating an account on GitHub." />

  <meta content="4267705" name="octolytics-dimension-user_id" /><meta content="rappoccio" name="octolytics-dimension-user_login" /><meta content="9702113" name="octolytics-dimension-repository_id" /><meta content="rappoccio/Exascale" name="octolytics-dimension-repository_nwo" /><meta content="true" name="octolytics-dimension-repository_public" /><meta content="false" name="octolytics-dimension-repository_is_fork" /><meta content="9702113" name="octolytics-dimension-repository_network_root_id" /><meta content="rappoccio/Exascale" name="octolytics-dimension-repository_network_root_nwo" />
  <link href="https://github.com/rappoccio/Exascale/commits/master.atom" rel="alternate" title="Recent Commits to Exascale:master" type="application/atom+xml" />

  </head>


  <body class="logged_out  env-production  vis-public  page-blob">
    <div class="wrapper">
      
      
      


      
      <div class="header header-logged-out">
  <div class="container clearfix">

    <a class="header-logo-wordmark" href="https://github.com/">
      <span class="mega-octicon octicon-logo-github"></span>
    </a>

    <div class="header-actions">
        <a class="button primary" href="/signup">Sign up</a>
      <a class="button signin" href="/login?return_to=%2Frappoccio%2FExascale%2Fblob%2Fmaster%2FCollaborativeFund2013%2Fcollaborative_proposal.tex">Sign in</a>
    </div>

    <div class="command-bar js-command-bar  in-repository">

      <ul class="top-nav">
          <li class="explore"><a href="/explore">Explore</a></li>
        <li class="features"><a href="/features">Features</a></li>
          <li class="enterprise"><a href="https://enterprise.github.com/">Enterprise</a></li>
          <li class="blog"><a href="/blog">Blog</a></li>
      </ul>
        <form accept-charset="UTF-8" action="/search" class="command-bar-form" id="top_search_form" method="get">

<input type="text" data-hotkey="/ s" name="q" id="js-command-bar-field" placeholder="Search or type a command" tabindex="1" autocapitalize="off"
    
    
      data-repo="rappoccio/Exascale"
      data-branch="master"
      data-sha="9f6c6f1526340ae8ea1329a46ce11cc0d31d7c74"
  >

    <input type="hidden" name="nwo" value="rappoccio/Exascale" />

    <div class="select-menu js-menu-container js-select-menu search-context-select-menu">
      <span class="minibutton select-menu-button js-menu-target">
        <span class="js-select-button">This repository</span>
      </span>

      <div class="select-menu-modal-holder js-menu-content js-navigation-container">
        <div class="select-menu-modal">

          <div class="select-menu-item js-navigation-item js-this-repository-navigation-item selected">
            <span class="select-menu-item-icon octicon octicon-check"></span>
            <input type="radio" class="js-search-this-repository" name="search_target" value="repository" checked="checked" />
            <div class="select-menu-item-text js-select-button-text">This repository</div>
          </div> <!-- /.select-menu-item -->

          <div class="select-menu-item js-navigation-item js-all-repositories-navigation-item">
            <span class="select-menu-item-icon octicon octicon-check"></span>
            <input type="radio" name="search_target" value="global" />
            <div class="select-menu-item-text js-select-button-text">All repositories</div>
          </div> <!-- /.select-menu-item -->

        </div>
      </div>
    </div>

  <span class="octicon help tooltipped downwards" title="Show command bar help">
    <span class="octicon octicon-question"></span>
  </span>


  <input type="hidden" name="ref" value="cmdform">

</form>
    </div>

  </div>
</div>


      


          <div class="site" itemscope itemtype="http://schema.org/WebPage">
    
    <div class="pagehead repohead instapaper_ignore readability-menu">
      <div class="container">
        

<ul class="pagehead-actions">


  <li>
  <a href="/login?return_to=%2Frappoccio%2FExascale"
  class="minibutton with-count js-toggler-target star-button entice tooltipped upwards"
  title="You must be signed in to use this feature" rel="nofollow">
  <span class="octicon octicon-star"></span>Star
</a>
<a class="social-count js-social-count" href="/rappoccio/Exascale/stargazers">
  0
</a>

  </li>

    <li>
      <a href="/login?return_to=%2Frappoccio%2FExascale"
        class="minibutton with-count js-toggler-target fork-button entice tooltipped upwards"
        title="You must be signed in to fork a repository" rel="nofollow">
        <span class="octicon octicon-git-branch"></span>Fork
      </a>
      <a href="/rappoccio/Exascale/network" class="social-count">
        0
      </a>
    </li>
</ul>

        <h1 itemscope itemtype="http://data-vocabulary.org/Breadcrumb" class="entry-title public">
          <span class="repo-label"><span>public</span></span>
          <span class="mega-octicon octicon-repo"></span>
          <span class="author">
            <a href="/rappoccio" class="url fn" itemprop="url" rel="author"><span itemprop="title">rappoccio</span></a></span
          ><span class="repohead-name-divider">/</span><strong
          ><a href="/rappoccio/Exascale" class="js-current-repository js-repo-home-link">Exascale</a></strong>

          <span class="page-context-loader">
            <img alt="Octocat-spinner-32" height="16" src="https://github.global.ssl.fastly.net/images/spinners/octocat-spinner-32.gif" width="16" />
          </span>

        </h1>
      </div><!-- /.container -->
    </div><!-- /.repohead -->

    <div class="container">

      <div class="repository-with-sidebar repo-container ">

        <div class="repository-sidebar">
            

<div class="repo-nav repo-nav-full js-repository-container-pjax js-octicon-loaders">
  <div class="repo-nav-contents">
    <ul class="repo-menu">
      <li class="tooltipped leftwards" title="Code">
        <a href="/rappoccio/Exascale" aria-label="Code" class="js-selected-navigation-item selected" data-gotokey="c" data-pjax="true" data-selected-links="repo_source repo_downloads repo_commits repo_tags repo_branches /rappoccio/Exascale">
          <span class="octicon octicon-code"></span> <span class="full-word">Code</span>
          <img alt="Octocat-spinner-32" class="mini-loader" height="16" src="https://github.global.ssl.fastly.net/images/spinners/octocat-spinner-32.gif" width="16" />
</a>      </li>

        <li class="tooltipped leftwards" title="Issues">
          <a href="/rappoccio/Exascale/issues" aria-label="Issues" class="js-selected-navigation-item js-disable-pjax" data-gotokey="i" data-selected-links="repo_issues /rappoccio/Exascale/issues">
            <span class="octicon octicon-issue-opened"></span> <span class="full-word">Issues</span>
            <span class='counter'>0</span>
            <img alt="Octocat-spinner-32" class="mini-loader" height="16" src="https://github.global.ssl.fastly.net/images/spinners/octocat-spinner-32.gif" width="16" />
</a>        </li>

      <li class="tooltipped leftwards" title="Pull Requests"><a href="/rappoccio/Exascale/pulls" aria-label="Pull Requests" class="js-selected-navigation-item js-disable-pjax" data-gotokey="p" data-selected-links="repo_pulls /rappoccio/Exascale/pulls">
            <span class="octicon octicon-git-pull-request"></span> <span class="full-word">Pull Requests</span>
            <span class='counter'>0</span>
            <img alt="Octocat-spinner-32" class="mini-loader" height="16" src="https://github.global.ssl.fastly.net/images/spinners/octocat-spinner-32.gif" width="16" />
</a>      </li>


    </ul>
    <div class="repo-menu-separator"></div>
    <ul class="repo-menu">

      <li class="tooltipped leftwards" title="Pulse">
        <a href="/rappoccio/Exascale/pulse" aria-label="Pulse" class="js-selected-navigation-item " data-pjax="true" data-selected-links="pulse /rappoccio/Exascale/pulse">
          <span class="octicon octicon-pulse"></span> <span class="full-word">Pulse</span>
          <img alt="Octocat-spinner-32" class="mini-loader" height="16" src="https://github.global.ssl.fastly.net/images/spinners/octocat-spinner-32.gif" width="16" />
</a>      </li>

      <li class="tooltipped leftwards" title="Graphs">
        <a href="/rappoccio/Exascale/graphs" aria-label="Graphs" class="js-selected-navigation-item " data-pjax="true" data-selected-links="repo_graphs repo_contributors /rappoccio/Exascale/graphs">
          <span class="octicon octicon-graph"></span> <span class="full-word">Graphs</span>
          <img alt="Octocat-spinner-32" class="mini-loader" height="16" src="https://github.global.ssl.fastly.net/images/spinners/octocat-spinner-32.gif" width="16" />
</a>      </li>

      <li class="tooltipped leftwards" title="Network">
        <a href="/rappoccio/Exascale/network" aria-label="Network" class="js-selected-navigation-item js-disable-pjax" data-selected-links="repo_network /rappoccio/Exascale/network">
          <span class="octicon octicon-git-branch"></span> <span class="full-word">Network</span>
          <img alt="Octocat-spinner-32" class="mini-loader" height="16" src="https://github.global.ssl.fastly.net/images/spinners/octocat-spinner-32.gif" width="16" />
</a>      </li>
    </ul>


  </div>
</div>

            <div class="only-with-full-nav">
              

  

<div class="clone-url open"
  data-protocol-type="http"
  data-url="/users/set_protocol?protocol_selector=http&amp;protocol_type=clone">
  <h3><strong>HTTPS</strong> clone URL</h3>
  <div class="clone-url-box">
    <input type="text" class="clone js-url-field"
           value="https://github.com/rappoccio/Exascale.git" readonly="readonly">

    <span class="js-zeroclipboard url-box-clippy minibutton zeroclipboard-button" data-clipboard-text="https://github.com/rappoccio/Exascale.git" data-copied-hint="copied!" title="copy to clipboard"><span class="octicon octicon-clippy"></span></span>
  </div>
</div>

  

<div class="clone-url "
  data-protocol-type="subversion"
  data-url="/users/set_protocol?protocol_selector=subversion&amp;protocol_type=clone">
  <h3><strong>Subversion</strong> checkout URL</h3>
  <div class="clone-url-box">
    <input type="text" class="clone js-url-field"
           value="https://github.com/rappoccio/Exascale" readonly="readonly">

    <span class="js-zeroclipboard url-box-clippy minibutton zeroclipboard-button" data-clipboard-text="https://github.com/rappoccio/Exascale" data-copied-hint="copied!" title="copy to clipboard"><span class="octicon octicon-clippy"></span></span>
  </div>
</div>


<p class="clone-options">You can clone with
      <a href="#" class="js-clone-selector" data-protocol="http">HTTPS</a>,
      or <a href="#" class="js-clone-selector" data-protocol="subversion">Subversion</a>.
  <span class="octicon help tooltipped upwards" title="Get help on which URL is right for you.">
    <a href="https://help.github.com/articles/which-remote-url-should-i-use">
    <span class="octicon octicon-question"></span>
    </a>
  </span>
</p>



                <a href="/rappoccio/Exascale/archive/master.zip"
                   class="minibutton sidebar-button"
                   title="Download this repository as a zip file"
                   rel="nofollow">
                  <span class="octicon octicon-cloud-download"></span>
                  Download ZIP
                </a>
            </div>
        </div><!-- /.repository-sidebar -->

        <div id="js-repo-pjax-container" class="repository-content context-loader-container" data-pjax-container>
          


<!-- blob contrib key: blob_contributors:v21:ba3e9dc57ad3f060a0ea4f159a8d31ae -->
<!-- blob contrib frag key: views10/v8/blob_contributors:v21:ba3e9dc57ad3f060a0ea4f159a8d31ae -->

<p title="This is a placeholder element" class="js-history-link-replace hidden"></p>

<a href="/rappoccio/Exascale/find/master" data-pjax data-hotkey="t" class="js-show-file-finder" style="display:none">Show File Finder</a>

<div class="file-navigation">
  


<div class="select-menu js-menu-container js-select-menu" >
  <span class="minibutton select-menu-button js-menu-target" data-hotkey="w"
    data-master-branch="master"
    data-ref="master"
    role="button" aria-label="Switch branches or tags" tabindex="0">
    <span class="octicon octicon-git-branch"></span>
    <i>branch:</i>
    <span class="js-select-button">master</span>
  </span>

  <div class="select-menu-modal-holder js-menu-content js-navigation-container" data-pjax>

    <div class="select-menu-modal">
      <div class="select-menu-header">
        <span class="select-menu-title">Switch branches/tags</span>
        <span class="octicon octicon-remove-close js-menu-close"></span>
      </div> <!-- /.select-menu-header -->

      <div class="select-menu-filters">
        <div class="select-menu-text-filter">
          <input type="text" aria-label="Filter branches/tags" id="context-commitish-filter-field" class="js-filterable-field js-navigation-enable" placeholder="Filter branches/tags">
        </div>
        <div class="select-menu-tabs">
          <ul>
            <li class="select-menu-tab">
              <a href="#" data-tab-filter="branches" class="js-select-menu-tab">Branches</a>
            </li>
            <li class="select-menu-tab">
              <a href="#" data-tab-filter="tags" class="js-select-menu-tab">Tags</a>
            </li>
          </ul>
        </div><!-- /.select-menu-tabs -->
      </div><!-- /.select-menu-filters -->

      <div class="select-menu-list select-menu-tab-bucket js-select-menu-tab-bucket" data-tab-filter="branches">

        <div data-filterable-for="context-commitish-filter-field" data-filterable-type="substring">


            <div class="select-menu-item js-navigation-item selected">
              <span class="select-menu-item-icon octicon octicon-check"></span>
              <a href="/rappoccio/Exascale/blob/master/CollaborativeFund2013/collaborative_proposal.tex"
                 data-name="master"
                 data-skip-pjax="true"
                 rel="nofollow"
                 class="js-navigation-open select-menu-item-text js-select-button-text css-truncate-target"
                 title="master">master</a>
            </div> <!-- /.select-menu-item -->
        </div>

          <div class="select-menu-no-results">Nothing to show</div>
      </div> <!-- /.select-menu-list -->

      <div class="select-menu-list select-menu-tab-bucket js-select-menu-tab-bucket" data-tab-filter="tags">
        <div data-filterable-for="context-commitish-filter-field" data-filterable-type="substring">


        </div>

        <div class="select-menu-no-results">Nothing to show</div>
      </div> <!-- /.select-menu-list -->

    </div> <!-- /.select-menu-modal -->
  </div> <!-- /.select-menu-modal-holder -->
</div> <!-- /.select-menu -->

  <div class="breadcrumb">
    <span class='repo-root js-repo-root'><span itemscope="" itemtype="http://data-vocabulary.org/Breadcrumb"><a href="/rappoccio/Exascale" data-branch="master" data-direction="back" data-pjax="true" itemscope="url"><span itemprop="title">Exascale</span></a></span></span><span class="separator"> / </span><span itemscope="" itemtype="http://data-vocabulary.org/Breadcrumb"><a href="/rappoccio/Exascale/tree/master/CollaborativeFund2013" data-branch="master" data-direction="back" data-pjax="true" itemscope="url"><span itemprop="title">CollaborativeFund2013</span></a></span><span class="separator"> / </span><strong class="final-path">collaborative_proposal.tex</strong> <span class="js-zeroclipboard minibutton zeroclipboard-button" data-clipboard-text="CollaborativeFund2013/collaborative_proposal.tex" data-copied-hint="copied!" title="copy to clipboard"><span class="octicon octicon-clippy"></span></span>
  </div>
</div>


  <div class="commit commit-loader file-history-tease js-deferred-content" data-url="/rappoccio/Exascale/contributors/master/CollaborativeFund2013/collaborative_proposal.tex">
    Fetching contributors…

    <div class="participation">
      <p class="loader-loading"><img alt="Octocat-spinner-32-eaf2f5" height="16" src="https://github.global.ssl.fastly.net/images/spinners/octocat-spinner-32-EAF2F5.gif" width="16" /></p>
      <p class="loader-error">Cannot retrieve contributors at this time</p>
    </div>
  </div>

<div id="files" class="bubble">
  <div class="file">
    <div class="meta">
      <div class="info">
        <span class="icon"><b class="octicon octicon-file-text"></b></span>
        <span class="mode" title="File Mode">file</span>
          <span>364 lines (304 sloc)</span>
        <span>17.234 kb</span>
      </div>
      <div class="actions">
        <div class="button-group">
              <a class="minibutton disabled js-entice" href=""
                 data-entice="You must be signed in to make or propose changes">Edit</a>
          <a href="/rappoccio/Exascale/raw/master/CollaborativeFund2013/collaborative_proposal.tex" class="button minibutton " id="raw-url">Raw</a>
            <a href="/rappoccio/Exascale/blame/master/CollaborativeFund2013/collaborative_proposal.tex" class="button minibutton ">Blame</a>
          <a href="/rappoccio/Exascale/commits/master/CollaborativeFund2013/collaborative_proposal.tex" class="button minibutton " rel="nofollow">History</a>
        </div><!-- /.button-group -->
            <a class="minibutton danger empty-icon js-entice" href=""
               data-entice="You must be signed in and on a branch to make or propose changes">
            Delete
          </a>
      </div><!-- /.actions -->

    </div>
        <div class="blob-wrapper data type-tex js-blob-data">
        <table class="file-code file-diff">
          <tr class="file-code-line">
            <td class="blob-line-nums">
              <span id="L1" rel="#L1">1</span>
<span id="L2" rel="#L2">2</span>
<span id="L3" rel="#L3">3</span>
<span id="L4" rel="#L4">4</span>
<span id="L5" rel="#L5">5</span>
<span id="L6" rel="#L6">6</span>
<span id="L7" rel="#L7">7</span>
<span id="L8" rel="#L8">8</span>
<span id="L9" rel="#L9">9</span>
<span id="L10" rel="#L10">10</span>
<span id="L11" rel="#L11">11</span>
<span id="L12" rel="#L12">12</span>
<span id="L13" rel="#L13">13</span>
<span id="L14" rel="#L14">14</span>
<span id="L15" rel="#L15">15</span>
<span id="L16" rel="#L16">16</span>
<span id="L17" rel="#L17">17</span>
<span id="L18" rel="#L18">18</span>
<span id="L19" rel="#L19">19</span>
<span id="L20" rel="#L20">20</span>
<span id="L21" rel="#L21">21</span>
<span id="L22" rel="#L22">22</span>
<span id="L23" rel="#L23">23</span>
<span id="L24" rel="#L24">24</span>
<span id="L25" rel="#L25">25</span>
<span id="L26" rel="#L26">26</span>
<span id="L27" rel="#L27">27</span>
<span id="L28" rel="#L28">28</span>
<span id="L29" rel="#L29">29</span>
<span id="L30" rel="#L30">30</span>
<span id="L31" rel="#L31">31</span>
<span id="L32" rel="#L32">32</span>
<span id="L33" rel="#L33">33</span>
<span id="L34" rel="#L34">34</span>
<span id="L35" rel="#L35">35</span>
<span id="L36" rel="#L36">36</span>
<span id="L37" rel="#L37">37</span>
<span id="L38" rel="#L38">38</span>
<span id="L39" rel="#L39">39</span>
<span id="L40" rel="#L40">40</span>
<span id="L41" rel="#L41">41</span>
<span id="L42" rel="#L42">42</span>
<span id="L43" rel="#L43">43</span>
<span id="L44" rel="#L44">44</span>
<span id="L45" rel="#L45">45</span>
<span id="L46" rel="#L46">46</span>
<span id="L47" rel="#L47">47</span>
<span id="L48" rel="#L48">48</span>
<span id="L49" rel="#L49">49</span>
<span id="L50" rel="#L50">50</span>
<span id="L51" rel="#L51">51</span>
<span id="L52" rel="#L52">52</span>
<span id="L53" rel="#L53">53</span>
<span id="L54" rel="#L54">54</span>
<span id="L55" rel="#L55">55</span>
<span id="L56" rel="#L56">56</span>
<span id="L57" rel="#L57">57</span>
<span id="L58" rel="#L58">58</span>
<span id="L59" rel="#L59">59</span>
<span id="L60" rel="#L60">60</span>
<span id="L61" rel="#L61">61</span>
<span id="L62" rel="#L62">62</span>
<span id="L63" rel="#L63">63</span>
<span id="L64" rel="#L64">64</span>
<span id="L65" rel="#L65">65</span>
<span id="L66" rel="#L66">66</span>
<span id="L67" rel="#L67">67</span>
<span id="L68" rel="#L68">68</span>
<span id="L69" rel="#L69">69</span>
<span id="L70" rel="#L70">70</span>
<span id="L71" rel="#L71">71</span>
<span id="L72" rel="#L72">72</span>
<span id="L73" rel="#L73">73</span>
<span id="L74" rel="#L74">74</span>
<span id="L75" rel="#L75">75</span>
<span id="L76" rel="#L76">76</span>
<span id="L77" rel="#L77">77</span>
<span id="L78" rel="#L78">78</span>
<span id="L79" rel="#L79">79</span>
<span id="L80" rel="#L80">80</span>
<span id="L81" rel="#L81">81</span>
<span id="L82" rel="#L82">82</span>
<span id="L83" rel="#L83">83</span>
<span id="L84" rel="#L84">84</span>
<span id="L85" rel="#L85">85</span>
<span id="L86" rel="#L86">86</span>
<span id="L87" rel="#L87">87</span>
<span id="L88" rel="#L88">88</span>
<span id="L89" rel="#L89">89</span>
<span id="L90" rel="#L90">90</span>
<span id="L91" rel="#L91">91</span>
<span id="L92" rel="#L92">92</span>
<span id="L93" rel="#L93">93</span>
<span id="L94" rel="#L94">94</span>
<span id="L95" rel="#L95">95</span>
<span id="L96" rel="#L96">96</span>
<span id="L97" rel="#L97">97</span>
<span id="L98" rel="#L98">98</span>
<span id="L99" rel="#L99">99</span>
<span id="L100" rel="#L100">100</span>
<span id="L101" rel="#L101">101</span>
<span id="L102" rel="#L102">102</span>
<span id="L103" rel="#L103">103</span>
<span id="L104" rel="#L104">104</span>
<span id="L105" rel="#L105">105</span>
<span id="L106" rel="#L106">106</span>
<span id="L107" rel="#L107">107</span>
<span id="L108" rel="#L108">108</span>
<span id="L109" rel="#L109">109</span>
<span id="L110" rel="#L110">110</span>
<span id="L111" rel="#L111">111</span>
<span id="L112" rel="#L112">112</span>
<span id="L113" rel="#L113">113</span>
<span id="L114" rel="#L114">114</span>
<span id="L115" rel="#L115">115</span>
<span id="L116" rel="#L116">116</span>
<span id="L117" rel="#L117">117</span>
<span id="L118" rel="#L118">118</span>
<span id="L119" rel="#L119">119</span>
<span id="L120" rel="#L120">120</span>
<span id="L121" rel="#L121">121</span>
<span id="L122" rel="#L122">122</span>
<span id="L123" rel="#L123">123</span>
<span id="L124" rel="#L124">124</span>
<span id="L125" rel="#L125">125</span>
<span id="L126" rel="#L126">126</span>
<span id="L127" rel="#L127">127</span>
<span id="L128" rel="#L128">128</span>
<span id="L129" rel="#L129">129</span>
<span id="L130" rel="#L130">130</span>
<span id="L131" rel="#L131">131</span>
<span id="L132" rel="#L132">132</span>
<span id="L133" rel="#L133">133</span>
<span id="L134" rel="#L134">134</span>
<span id="L135" rel="#L135">135</span>
<span id="L136" rel="#L136">136</span>
<span id="L137" rel="#L137">137</span>
<span id="L138" rel="#L138">138</span>
<span id="L139" rel="#L139">139</span>
<span id="L140" rel="#L140">140</span>
<span id="L141" rel="#L141">141</span>
<span id="L142" rel="#L142">142</span>
<span id="L143" rel="#L143">143</span>
<span id="L144" rel="#L144">144</span>
<span id="L145" rel="#L145">145</span>
<span id="L146" rel="#L146">146</span>
<span id="L147" rel="#L147">147</span>
<span id="L148" rel="#L148">148</span>
<span id="L149" rel="#L149">149</span>
<span id="L150" rel="#L150">150</span>
<span id="L151" rel="#L151">151</span>
<span id="L152" rel="#L152">152</span>
<span id="L153" rel="#L153">153</span>
<span id="L154" rel="#L154">154</span>
<span id="L155" rel="#L155">155</span>
<span id="L156" rel="#L156">156</span>
<span id="L157" rel="#L157">157</span>
<span id="L158" rel="#L158">158</span>
<span id="L159" rel="#L159">159</span>
<span id="L160" rel="#L160">160</span>
<span id="L161" rel="#L161">161</span>
<span id="L162" rel="#L162">162</span>
<span id="L163" rel="#L163">163</span>
<span id="L164" rel="#L164">164</span>
<span id="L165" rel="#L165">165</span>
<span id="L166" rel="#L166">166</span>
<span id="L167" rel="#L167">167</span>
<span id="L168" rel="#L168">168</span>
<span id="L169" rel="#L169">169</span>
<span id="L170" rel="#L170">170</span>
<span id="L171" rel="#L171">171</span>
<span id="L172" rel="#L172">172</span>
<span id="L173" rel="#L173">173</span>
<span id="L174" rel="#L174">174</span>
<span id="L175" rel="#L175">175</span>
<span id="L176" rel="#L176">176</span>
<span id="L177" rel="#L177">177</span>
<span id="L178" rel="#L178">178</span>
<span id="L179" rel="#L179">179</span>
<span id="L180" rel="#L180">180</span>
<span id="L181" rel="#L181">181</span>
<span id="L182" rel="#L182">182</span>
<span id="L183" rel="#L183">183</span>
<span id="L184" rel="#L184">184</span>
<span id="L185" rel="#L185">185</span>
<span id="L186" rel="#L186">186</span>
<span id="L187" rel="#L187">187</span>
<span id="L188" rel="#L188">188</span>
<span id="L189" rel="#L189">189</span>
<span id="L190" rel="#L190">190</span>
<span id="L191" rel="#L191">191</span>
<span id="L192" rel="#L192">192</span>
<span id="L193" rel="#L193">193</span>
<span id="L194" rel="#L194">194</span>
<span id="L195" rel="#L195">195</span>
<span id="L196" rel="#L196">196</span>
<span id="L197" rel="#L197">197</span>
<span id="L198" rel="#L198">198</span>
<span id="L199" rel="#L199">199</span>
<span id="L200" rel="#L200">200</span>
<span id="L201" rel="#L201">201</span>
<span id="L202" rel="#L202">202</span>
<span id="L203" rel="#L203">203</span>
<span id="L204" rel="#L204">204</span>
<span id="L205" rel="#L205">205</span>
<span id="L206" rel="#L206">206</span>
<span id="L207" rel="#L207">207</span>
<span id="L208" rel="#L208">208</span>
<span id="L209" rel="#L209">209</span>
<span id="L210" rel="#L210">210</span>
<span id="L211" rel="#L211">211</span>
<span id="L212" rel="#L212">212</span>
<span id="L213" rel="#L213">213</span>
<span id="L214" rel="#L214">214</span>
<span id="L215" rel="#L215">215</span>
<span id="L216" rel="#L216">216</span>
<span id="L217" rel="#L217">217</span>
<span id="L218" rel="#L218">218</span>
<span id="L219" rel="#L219">219</span>
<span id="L220" rel="#L220">220</span>
<span id="L221" rel="#L221">221</span>
<span id="L222" rel="#L222">222</span>
<span id="L223" rel="#L223">223</span>
<span id="L224" rel="#L224">224</span>
<span id="L225" rel="#L225">225</span>
<span id="L226" rel="#L226">226</span>
<span id="L227" rel="#L227">227</span>
<span id="L228" rel="#L228">228</span>
<span id="L229" rel="#L229">229</span>
<span id="L230" rel="#L230">230</span>
<span id="L231" rel="#L231">231</span>
<span id="L232" rel="#L232">232</span>
<span id="L233" rel="#L233">233</span>
<span id="L234" rel="#L234">234</span>
<span id="L235" rel="#L235">235</span>
<span id="L236" rel="#L236">236</span>
<span id="L237" rel="#L237">237</span>
<span id="L238" rel="#L238">238</span>
<span id="L239" rel="#L239">239</span>
<span id="L240" rel="#L240">240</span>
<span id="L241" rel="#L241">241</span>
<span id="L242" rel="#L242">242</span>
<span id="L243" rel="#L243">243</span>
<span id="L244" rel="#L244">244</span>
<span id="L245" rel="#L245">245</span>
<span id="L246" rel="#L246">246</span>
<span id="L247" rel="#L247">247</span>
<span id="L248" rel="#L248">248</span>
<span id="L249" rel="#L249">249</span>
<span id="L250" rel="#L250">250</span>
<span id="L251" rel="#L251">251</span>
<span id="L252" rel="#L252">252</span>
<span id="L253" rel="#L253">253</span>
<span id="L254" rel="#L254">254</span>
<span id="L255" rel="#L255">255</span>
<span id="L256" rel="#L256">256</span>
<span id="L257" rel="#L257">257</span>
<span id="L258" rel="#L258">258</span>
<span id="L259" rel="#L259">259</span>
<span id="L260" rel="#L260">260</span>
<span id="L261" rel="#L261">261</span>
<span id="L262" rel="#L262">262</span>
<span id="L263" rel="#L263">263</span>
<span id="L264" rel="#L264">264</span>
<span id="L265" rel="#L265">265</span>
<span id="L266" rel="#L266">266</span>
<span id="L267" rel="#L267">267</span>
<span id="L268" rel="#L268">268</span>
<span id="L269" rel="#L269">269</span>
<span id="L270" rel="#L270">270</span>
<span id="L271" rel="#L271">271</span>
<span id="L272" rel="#L272">272</span>
<span id="L273" rel="#L273">273</span>
<span id="L274" rel="#L274">274</span>
<span id="L275" rel="#L275">275</span>
<span id="L276" rel="#L276">276</span>
<span id="L277" rel="#L277">277</span>
<span id="L278" rel="#L278">278</span>
<span id="L279" rel="#L279">279</span>
<span id="L280" rel="#L280">280</span>
<span id="L281" rel="#L281">281</span>
<span id="L282" rel="#L282">282</span>
<span id="L283" rel="#L283">283</span>
<span id="L284" rel="#L284">284</span>
<span id="L285" rel="#L285">285</span>
<span id="L286" rel="#L286">286</span>
<span id="L287" rel="#L287">287</span>
<span id="L288" rel="#L288">288</span>
<span id="L289" rel="#L289">289</span>
<span id="L290" rel="#L290">290</span>
<span id="L291" rel="#L291">291</span>
<span id="L292" rel="#L292">292</span>
<span id="L293" rel="#L293">293</span>
<span id="L294" rel="#L294">294</span>
<span id="L295" rel="#L295">295</span>
<span id="L296" rel="#L296">296</span>
<span id="L297" rel="#L297">297</span>
<span id="L298" rel="#L298">298</span>
<span id="L299" rel="#L299">299</span>
<span id="L300" rel="#L300">300</span>
<span id="L301" rel="#L301">301</span>
<span id="L302" rel="#L302">302</span>
<span id="L303" rel="#L303">303</span>
<span id="L304" rel="#L304">304</span>
<span id="L305" rel="#L305">305</span>
<span id="L306" rel="#L306">306</span>
<span id="L307" rel="#L307">307</span>
<span id="L308" rel="#L308">308</span>
<span id="L309" rel="#L309">309</span>
<span id="L310" rel="#L310">310</span>
<span id="L311" rel="#L311">311</span>
<span id="L312" rel="#L312">312</span>
<span id="L313" rel="#L313">313</span>
<span id="L314" rel="#L314">314</span>
<span id="L315" rel="#L315">315</span>
<span id="L316" rel="#L316">316</span>
<span id="L317" rel="#L317">317</span>
<span id="L318" rel="#L318">318</span>
<span id="L319" rel="#L319">319</span>
<span id="L320" rel="#L320">320</span>
<span id="L321" rel="#L321">321</span>
<span id="L322" rel="#L322">322</span>
<span id="L323" rel="#L323">323</span>
<span id="L324" rel="#L324">324</span>
<span id="L325" rel="#L325">325</span>
<span id="L326" rel="#L326">326</span>
<span id="L327" rel="#L327">327</span>
<span id="L328" rel="#L328">328</span>
<span id="L329" rel="#L329">329</span>
<span id="L330" rel="#L330">330</span>
<span id="L331" rel="#L331">331</span>
<span id="L332" rel="#L332">332</span>
<span id="L333" rel="#L333">333</span>
<span id="L334" rel="#L334">334</span>
<span id="L335" rel="#L335">335</span>
<span id="L336" rel="#L336">336</span>
<span id="L337" rel="#L337">337</span>
<span id="L338" rel="#L338">338</span>
<span id="L339" rel="#L339">339</span>
<span id="L340" rel="#L340">340</span>
<span id="L341" rel="#L341">341</span>
<span id="L342" rel="#L342">342</span>
<span id="L343" rel="#L343">343</span>
<span id="L344" rel="#L344">344</span>
<span id="L345" rel="#L345">345</span>
<span id="L346" rel="#L346">346</span>
<span id="L347" rel="#L347">347</span>
<span id="L348" rel="#L348">348</span>
<span id="L349" rel="#L349">349</span>
<span id="L350" rel="#L350">350</span>
<span id="L351" rel="#L351">351</span>
<span id="L352" rel="#L352">352</span>
<span id="L353" rel="#L353">353</span>
<span id="L354" rel="#L354">354</span>
<span id="L355" rel="#L355">355</span>
<span id="L356" rel="#L356">356</span>
<span id="L357" rel="#L357">357</span>
<span id="L358" rel="#L358">358</span>
<span id="L359" rel="#L359">359</span>
<span id="L360" rel="#L360">360</span>
<span id="L361" rel="#L361">361</span>
<span id="L362" rel="#L362">362</span>
<span id="L363" rel="#L363">363</span>

            </td>
            <td class="blob-line-code">
                    <div class="highlight"><pre><div class='line' id='LC1'><span class="k">\documentclass</span><span class="na">[12pt]</span><span class="nb">{</span>article<span class="nb">}</span></div><div class='line' id='LC2'><span class="k">\usepackage</span><span class="na">[usenames]</span><span class="nb">{</span>color<span class="nb">}</span> <span class="c">%used for font color</span></div><div class='line' id='LC3'><span class="k">\usepackage</span><span class="nb">{</span>amssymb<span class="nb">}</span> <span class="c">%maths</span></div><div class='line' id='LC4'><span class="k">\usepackage</span><span class="nb">{</span>amsmath<span class="nb">}</span> <span class="c">%maths</span></div><div class='line' id='LC5'><span class="k">\usepackage</span><span class="nb">{</span>graphicx<span class="nb">}</span></div><div class='line' id='LC6'><span class="k">\usepackage</span><span class="nb">{</span>url<span class="nb">}</span></div><div class='line' id='LC7'><br/></div><div class='line' id='LC8'><span class="k">\newcommand</span><span class="nb">{</span><span class="k">\GeV</span><span class="nb">}{</span><span class="k">\ensuremath</span><span class="nb">{</span><span class="k">\mathrm</span><span class="nb">{</span>GeV<span class="nb">}}}</span></div><div class='line' id='LC9'><span class="k">\newcommand</span><span class="nb">{</span><span class="k">\TeV</span><span class="nb">}{</span><span class="k">\ensuremath</span><span class="nb">{</span><span class="k">\mathrm</span><span class="nb">{</span>TeV<span class="nb">}}}</span></div><div class='line' id='LC10'><span class="k">\newcommand</span><span class="nb">{</span><span class="k">\GeVc</span><span class="nb">}{</span><span class="k">\ensuremath</span><span class="nb">{</span><span class="k">\mathrm</span><span class="nb">{</span>GeV<span class="nb">}}}</span></div><div class='line' id='LC11'><span class="k">\newcommand</span><span class="nb">{</span><span class="k">\TeVc</span><span class="nb">}{</span><span class="k">\ensuremath</span><span class="nb">{</span><span class="k">\mathrm</span><span class="nb">{</span>TeV<span class="nb">}}}</span></div><div class='line' id='LC12'><span class="k">\newcommand</span><span class="nb">{</span><span class="k">\GeVcc</span><span class="nb">}{</span><span class="k">\ensuremath</span><span class="nb">{</span><span class="k">\mathrm</span><span class="nb">{</span>GeV<span class="nb">}}}</span></div><div class='line' id='LC13'><span class="k">\newcommand</span><span class="nb">{</span><span class="k">\TeVcc</span><span class="nb">}{</span><span class="k">\ensuremath</span><span class="nb">{</span><span class="k">\mathrm</span><span class="nb">{</span>TeV<span class="nb">}}}</span></div><div class='line' id='LC14'><span class="k">\newcommand</span><span class="nb">{</span><span class="k">\pt</span><span class="nb">}</span>            <span class="nb">{</span><span class="k">\ensuremath</span><span class="nb">{</span>p<span class="nb">_{</span><span class="k">\mathrm</span><span class="nb">{</span>T<span class="nb">}}}</span><span class="k">\xspace</span><span class="nb">}</span></div><div class='line' id='LC15'><span class="k">\newcommand</span><span class="nb">{</span><span class="k">\kt</span><span class="nb">}</span>            <span class="nb">{</span><span class="k">\ensuremath</span><span class="nb">{</span>k<span class="nb">_{</span><span class="k">\mathrm</span><span class="nb">{</span>T<span class="nb">}}}</span><span class="k">\xspace</span><span class="nb">}</span></div><div class='line' id='LC16'><span class="k">\newcommand</span><span class="nb">{</span><span class="k">\antikt</span><span class="nb">}</span>        <span class="nb">{</span>anti-<span class="k">\kt</span><span class="nb">}</span></div><div class='line' id='LC17'><span class="k">\newcommand</span><span class="nb">{</span><span class="k">\ttbar</span><span class="nb">}</span>        <span class="nb">{</span><span class="k">\ensuremath</span><span class="nb">{</span><span class="k">\mathrm</span><span class="nb">{</span>t<span class="nb">}</span><span class="k">\overline</span><span class="nb">{</span><span class="k">\mathrm</span><span class="nb">{</span>t<span class="nb">}}}}</span></div><div class='line' id='LC18'><span class="k">\newcommand</span><span class="nb">{</span><span class="k">\bbbar</span><span class="nb">}</span>        <span class="nb">{</span><span class="k">\ensuremath</span><span class="nb">{</span><span class="k">\mathrm</span><span class="nb">{</span>b<span class="nb">}</span><span class="k">\overline</span><span class="nb">{</span><span class="k">\mathrm</span><span class="nb">{</span>b<span class="nb">}}}}</span></div><div class='line' id='LC19'><span class="k">\newcommand</span><span class="nb">{</span><span class="k">\qqbar</span><span class="nb">}</span>        <span class="nb">{</span><span class="k">\ensuremath</span><span class="nb">{</span><span class="k">\mathrm</span><span class="nb">{</span>q<span class="nb">}</span><span class="k">\overline</span><span class="nb">{</span><span class="k">\mathrm</span><span class="nb">{</span>q<span class="nb">}}}}</span></div><div class='line' id='LC20'><span class="k">\newcommand</span><span class="nb">{</span><span class="k">\fbinv</span><span class="nb">}</span>        <span class="nb">{</span><span class="k">\ensuremath</span><span class="nb">{</span><span class="k">\mathrm</span><span class="nb">{</span>fb<span class="nb">}^{</span>-1<span class="nb">}}}</span></div><div class='line' id='LC21'><span class="k">\newcommand</span><span class="nb">{</span><span class="k">\instlumiA</span><span class="nb">}</span>     <span class="nb">{</span><span class="k">\ensuremath</span><span class="nb">{</span><span class="k">\times</span> 10<span class="nb">^{</span>33<span class="nb">}</span> <span class="k">\,\mathrm</span><span class="nb">{</span>cm<span class="nb">}^{</span>-2<span class="nb">}</span> <span class="k">\mathrm</span><span class="nb">{</span>s<span class="nb">}^{</span>-1<span class="nb">}}}</span></div><div class='line' id='LC22'><span class="k">\newcommand</span><span class="nb">{</span><span class="k">\instlumiB</span><span class="nb">}</span>     <span class="nb">{</span><span class="k">\ensuremath</span><span class="nb">{</span><span class="k">\times</span> 10<span class="nb">^{</span>34<span class="nb">}</span> <span class="k">\,\mathrm</span><span class="nb">{</span>cm<span class="nb">}^{</span>-2<span class="nb">}</span> <span class="k">\mathrm</span><span class="nb">{</span>s<span class="nb">}^{</span>-1<span class="nb">}}}</span></div><div class='line' id='LC23'><br/></div><div class='line' id='LC24'><span class="k">\begin</span><span class="nb">{</span>document<span class="nb">}</span></div><div class='line' id='LC25'><br/></div><div class='line' id='LC26'><span class="k">\title</span><span class="nb">{</span>Exascale Computing at the LHC : Narrative<span class="nb">}</span></div><div class='line' id='LC27'><span class="k">\author</span><span class="nb">{{</span><span class="k">\bf</span> Principle Investigators<span class="nb">}</span> : <span class="k">\\</span>  </div><div class='line' id='LC28'>&nbsp;&nbsp;Reneta Barneva, Matthew Jones, <span class="k">\\</span> </div><div class='line' id='LC29'>&nbsp;&nbsp;Steven Ko, Salvatore Rappoccio, Lukasz Ziarek <span class="k">\\</span> <span class="k">\\</span> </div><div class='line' id='LC30'>&nbsp;&nbsp;<span class="nb">{</span><span class="k">\bf</span> Mentors<span class="nb">}</span> : Valentin Brimkov, Peter Elmer<span class="nb">}</span></div><div class='line' id='LC31'><br/></div><div class='line' id='LC32'><span class="k">\maketitle</span></div><div class='line' id='LC33'><br/></div><div class='line' id='LC34'><span class="k">\clearpage</span></div><div class='line' id='LC35'><br/></div><div class='line' id='LC36'><span class="k">\section</span><span class="nb">{</span>Funding Strategy<span class="nb">}</span></div><div class='line' id='LC37'><br/></div><div class='line' id='LC38'>This research proposal outlines the necessity to extend the computing</div><div class='line' id='LC39'>capacity of the experiments at the Large Hadron Collider (LHC) in</div><div class='line' id='LC40'>Geneva, Switzerland, to the exascale of high-throughput</div><div class='line' id='LC41'>processing. This is an enormously challenging task, but a necessary</div><div class='line' id='LC42'>one to ensure the long-term success of the LHC experiments. </div><div class='line' id='LC43'><br/></div><div class='line' id='LC44'>Exascale computing (in this case, in high throughput)</div><div class='line' id='LC45'>is an enormously growth-oriented area. Even during</div><div class='line' id='LC46'>lean economic times, the US Federal Government is pledging to support</div><div class='line' id='LC47'>this area of research, for instance in the Department of Energy</div><div class='line' id='LC48'>Exascale Computing Initiative~<span class="k">\cite</span><span class="nb">{</span>doe<span class="nb">_</span>eci<span class="nb">}</span>. Indeed, the DOE Office of</div><div class='line' id='LC49'>Science quotes :</div><div class='line' id='LC50'><span class="k">\begin</span><span class="nb">{</span>quote<span class="nb">}</span></div><div class='line' id='LC51'>The Exascale initiative will be significant and transformative for Department of Energy missions.</div><div class='line' id='LC52'><span class="k">\end</span><span class="nb">{</span>quote<span class="nb">}</span></div><div class='line' id='LC53'>The current level of funding for the DOE EIC and related activities is</div><div class='line' id='LC54'><span class="k">\$</span>21M.~<span class="k">\cite</span><span class="nb">{</span>doe<span class="nb">_</span>eci<span class="nb">_</span>budget<span class="nb">}</span>. This ``transformative&#39;&#39; strategy is one</div><div class='line' id='LC55'>that is envisioned to continue well into the future. </div><div class='line' id='LC56'>In addition to DOE programs, the</div><div class='line' id='LC57'>National Science Foundation (NSF) has several programs to address the</div><div class='line' id='LC58'>problems of high-throughput exascale computing~<span class="k">\cite</span><span class="nb">{</span>nsf1,nsf2<span class="nb">}</span>.</div><div class='line' id='LC59'><br/></div><div class='line' id='LC60'>In addition to governmental programs such as above, private sector</div><div class='line' id='LC61'>funding sources are also available, such as the Google Faculty</div><div class='line' id='LC62'>Research Awards~<span class="k">\cite</span><span class="nb">{</span>google<span class="nb">_</span>fac<span class="nb">_</span>awards<span class="nb">}</span>, which has an interest in</div><div class='line' id='LC63'>exascale computing projects also. </div><div class='line' id='LC64'><br/></div><div class='line' id='LC65'>We plan to apply for these projects in the coming year 2013-2014. </div><div class='line' id='LC66'><br/></div><div class='line' id='LC67'><br/></div><div class='line' id='LC68'><span class="k">\clearpage</span></div><div class='line' id='LC69'><br/></div><div class='line' id='LC70'><span class="k">\section</span><span class="nb">{</span>Project Organization<span class="nb">}</span></div><div class='line' id='LC71'><br/></div><div class='line' id='LC72'>The principle investigators (PIs) of this proposal have a</div><div class='line' id='LC73'>widely-varied and applicable skill set to accomplish the goals of</div><div class='line' id='LC74'>extending LHC computing to the exascale. </div><div class='line' id='LC75'><br/></div><div class='line' id='LC76'><span class="k">\begin</span><span class="nb">{</span>itemize<span class="nb">}</span></div><div class='line' id='LC77'><span class="k">\item</span> Salvatore Rappoccio has 15 years of experience programming in a</div><div class='line' id='LC78'>high-energy physics environment, as well as other numerical software</div><div class='line' id='LC79'>design for the private sector. He is an expert in critical</div><div class='line' id='LC80'>areas of event reconstruction at CMS which can be optimized for</div><div class='line' id='LC81'>multicore usage. </div><div class='line' id='LC82'><span class="k">\item</span> Lukasz Ziarek has 9 years of experience in language, compiler, and runtime design</div><div class='line' id='LC83'>targeted at improving multicore performance.  He has worked on 5 compilers and 3 Java VMs. He</div><div class='line' id='LC84'>is an expert at speculative and transactional computation focusing on the extraction of</div><div class='line' id='LC85'>parallelism and lightweight concurrency.</div><div class='line' id='LC86'><span class="k">\item</span> Steven Ko has 10 years of experience in distributed systems. His recent focus has been large-scale data processing in the cloud using MapReduce and other technologies built on top of it. He also has 5 years of experience in large-scale storage and data management in data centers.</div><div class='line' id='LC87'><span class="k">\item</span> Matthew Jones has more than 12 years experience in scientific software development, </div><div class='line' id='LC88'>with a particular emphasis on parallel programming.</div><div class='line' id='LC89'>As the Associate Director of the Center for Computational Research at the University at Buffalo, </div><div class='line' id='LC90'>he has also been responsible for designing </div><div class='line' id='LC91'>and administering high-performance computing facilites for large-scale</div><div class='line' id='LC92'>scientific computing.</div><div class='line' id='LC93'><span class="k">\item</span> Reneta Barneva has 29 years of experience as a computer</div><div class='line' id='LC94'>&nbsp;&nbsp;scientist, and is the author of over 50 publications in the subjects</div><div class='line' id='LC95'>&nbsp;&nbsp;of combinatorial image analysis, pattern recognition, and</div><div class='line' id='LC96'>&nbsp;&nbsp;computational geometry. </div><div class='line' id='LC97'><span class="k">\end</span><span class="nb">{</span>itemize<span class="nb">}</span></div><div class='line' id='LC98'><br/></div><div class='line' id='LC99'>There are also two senior mentors to the project to offer guidance,</div><div class='line' id='LC100'>insight and technical information. </div><div class='line' id='LC101'><span class="k">\begin</span><span class="nb">{</span>itemize<span class="nb">}</span></div><div class='line' id='LC102'><span class="k">\item</span> Valentin Brimkov has over 20 years of experience as a</div><div class='line' id='LC103'>&nbsp;&nbsp;mathematician, and has extensive expertise</div><div class='line' id='LC104'>&nbsp;&nbsp;in design and analysis of algorithms, combinatorial optimization,</div><div class='line' id='LC105'>&nbsp;&nbsp;and discrete geometry. </div><div class='line' id='LC106'><span class="k">\item</span> Peter Elmer is the deputy offline software coordinator of the CMS</div><div class='line' id='LC107'>experiment at the LHC. He was responsible for the design and implementation</div><div class='line' id='LC108'>of the data and workflow management system used by CMS, as well as its core</div><div class='line' id='LC109'>event processing software and software development environment. He has</div><div class='line' id='LC110'>made important contributions to the software and computing of several </div><div class='line' id='LC111'>high-energy physics experiments over the past 20 years and is currently</div><div class='line' id='LC112'>focused on planning the computing R<span class="k">\&amp;</span>D needed for the next decade in CMS.</div><div class='line' id='LC113'><br/></div><div class='line' id='LC114'><span class="k">\end</span><span class="nb">{</span>itemize<span class="nb">}</span></div><div class='line' id='LC115'><br/></div><div class='line' id='LC116'>In addition, the proposal includes funding for</div><div class='line' id='LC117'>one graduate student from the CSE department at SUNY Buffalo at 100<span class="k">\%</span></div><div class='line' id='LC118'>FTE, and one graduate student from the Physics Department at SUNY</div><div class='line' id='LC119'>Buffalo at 50<span class="k">\%</span> FTE. They will be performing the study of this problem</div><div class='line' id='LC120'>under the direction of the PIs, and will also be responsible for</div><div class='line' id='LC121'>deployment and scalability of the solution. </div><div class='line' id='LC122'><br/></div><div class='line' id='LC123'><span class="k">\clearpage</span></div><div class='line' id='LC124'><br/></div><div class='line' id='LC125'><span class="k">\section</span><span class="nb">{</span>Narrative<span class="nb">}</span></div><div class='line' id='LC126'><br/></div><div class='line' id='LC127'>With the discovery of the Higgs boson </div><div class='line' id='LC128'>by the Large Hadron Collider (LHC)</div><div class='line' id='LC129'>experiments ATLAS and CMS~<span class="k">\cite</span><span class="nb">{</span>higgs<span class="nb">_</span>cms,higgs<span class="nb">_</span>atlas<span class="nb">}</span>, the</div><div class='line' id='LC130'>standard model (SM) of particle</div><div class='line' id='LC131'>physics is now complete. This model unifies the electromagnetic force</div><div class='line' id='LC132'>(carried by the <span class="nb">{</span><span class="k">\em</span> photon<span class="nb">}</span>) with the weak force, responsible for</div><div class='line' id='LC133'>radioactive decay (carried by the <span class="nb">{</span><span class="k">\em</span> <span class="s">$</span><span class="nb">W</span><span class="s">$</span> and <span class="s">$</span><span class="nb">Z</span><span class="s">$</span> bosons<span class="nb">}</span>).</div><div class='line' id='LC134'>At long last,</div><div class='line' id='LC135'>physicists now understand that via interactions with the Higgs field,</div><div class='line' id='LC136'>the <span class="s">$</span><span class="nb">W</span><span class="s">$</span> and <span class="s">$</span><span class="nb">Z</span><span class="s">$</span> bosons acquire a mass, but the photon does not. </div><div class='line' id='LC137'>This is referred to as ``electroweak symmetry breaking&#39;&#39;. </div><div class='line' id='LC138'><br/></div><div class='line' id='LC139'><br/></div><div class='line' id='LC140'>A new phase of particle physics has therefore begun. </div><div class='line' id='LC141'>The questions have shifted from the cause of electroweak symmetry</div><div class='line' id='LC142'>breaking, to the study of the Higgs boson and its interactions in</div><div class='line' id='LC143'>detail. </div><div class='line' id='LC144'>To understand the larger picture of the fundamental forces in nature,</div><div class='line' id='LC145'>the past excellence of the LHC experiments must therefore continue</div><div class='line' id='LC146'>unabated in the face of new technical challenges. </div><div class='line' id='LC147'><br/></div><div class='line' id='LC148'><br/></div><div class='line' id='LC149'>One of the major technical challenges that lies ahead is the</div><div class='line' id='LC150'>continuation of the scaling of computational power year by year, known</div><div class='line' id='LC151'>colloquially as ``Moore&#39;s Law&#39;&#39;. </div><div class='line' id='LC152'>To set the scale, at the CMS experiment</div><div class='line' id='LC153'>with the LHC collision flux (``luminosity&#39;&#39;) reaching</div><div class='line' id='LC154'><span class="s">$</span><span class="m">7</span><span class="nv">\instlumiA</span><span class="s">$</span>, the processing</div><div class='line' id='LC155'>time to reconstruct each collision event by CMS was approximately 20</div><div class='line' id='LC156'>seconds per event. However, as the luminosity is</div><div class='line' id='LC157'>increased, the computational time currently scales quadratically. As</div><div class='line' id='LC158'>the upgraded LHC is expected to deliver <span class="s">$</span><span class="nb">&gt;</span><span class="m">12</span><span class="nv">\instlumiA</span><span class="s">$</span> in the</div><div class='line' id='LC159'>upcoming run, the processing time per event is expected to reach</div><div class='line' id='LC160'>several minutes per event as shown in</div><div class='line' id='LC161'>Figure~<span class="k">\ref</span><span class="nb">{</span>lumitpeSingleMu<span class="nb">}</span>. Furthermore, in future runs of the LHC</div><div class='line' id='LC162'>in the next 15 years, the luminosity is expected to reach as high as</div><div class='line' id='LC163'><span class="s">$</span><span class="nb">&gt;</span><span class="m">1</span><span class="nv">\instlumiB</span><span class="s">$</span>, which would correspond (naively) to several hours of</div><div class='line' id='LC164'>computational time per event! Clearly, it is necessary for the</div><div class='line' id='LC165'>computing power to scale in order to compensate for this dramatic</div><div class='line' id='LC166'>increase in CPU time with increasing luminosity. </div><div class='line' id='LC167'><br/></div><div class='line' id='LC168'><br/></div><div class='line' id='LC169'><br/></div><div class='line' id='LC170'>However, with the expected end of the historic scaling of single-core</div><div class='line' id='LC171'>processing capability~<span class="k">\cite</span><span class="nb">{</span>GAMEOVER<span class="nb">}</span>, it is imperative to utilize a</div><div class='line' id='LC172'>parallel processing</div><div class='line' id='LC173'>strategy in order to maintain the levels of computational speed that</div><div class='line' id='LC174'>are required.</div><div class='line' id='LC175'><br/></div><div class='line' id='LC176'>Oftentimes,</div><div class='line' id='LC177'>the codes used by CMS (and experimental HEP in general) tend to lack</div><div class='line' id='LC178'>clear numerical ``kernels&#39;&#39; where optimization efforts can be focused. </div><div class='line' id='LC179'>Given these characteristics they are generally more properly classified as</div><div class='line' id='LC180'>``high throughput computing&#39;&#39; (HTC) rather than ``high performance computing&#39;&#39; (HPC). </div><div class='line' id='LC181'>In terms of their detailed behavior on the CPU many of these codes resemble</div><div class='line' id='LC182'>more general enterprise or ``cloud&#39;&#39;</div><div class='line' id='LC183'>applications~<span class="k">\cite</span><span class="nb">{</span>CLOUDSUITE,GOODACHEP<span class="nb">}</span>.</div><div class='line' id='LC184'><br/></div><div class='line' id='LC185'>However, there are several numerical algorithms where parallelization</div><div class='line' id='LC186'>could be exploited more directly. One of these is the so-called ``jet</div><div class='line' id='LC187'>clustering&#39;&#39; algorithms used at CMS, which we now discuss in detail. </div><div class='line' id='LC188'><span class="c">%At CMS, this computation is done very often by individuals rather than</span></div><div class='line' id='LC189'><span class="c">%centrally, and so is accounted for</span></div><div class='line' id='LC190'><span class="c">%in the 40\% of CPU usage from ``mixed user analysis applications&#39;&#39; as</span></div><div class='line' id='LC191'><span class="c">%described above. </span></div><div class='line' id='LC192'><span class="c">%It is likely, therefore, that improvements observed</span></div><div class='line' id='LC193'><span class="c">%in jet clustering will primarily benefit this portion of the CPU</span></div><div class='line' id='LC194'><span class="c">%usage. </span></div><div class='line' id='LC195'><span class="c">%We now discuss the prospects for</span></div><div class='line' id='LC196'><span class="c">%utilizing parallelization in jet clustering in detail. </span></div><div class='line' id='LC197'><br/></div><div class='line' id='LC198'>&nbsp;</div><div class='line' id='LC199'><br/></div><div class='line' id='LC200'><span class="k">\subsection</span><span class="nb">{</span>Jet Clustering<span class="nb">}</span> </div><div class='line' id='LC201'><br/></div><div class='line' id='LC202'><br/></div><div class='line' id='LC203'><br/></div><div class='line' id='LC204'><br/></div><div class='line' id='LC205'>The energetic deposits of particles in detectors need to be clustered</div><div class='line' id='LC206'>to obtain the</div><div class='line' id='LC207'>complete response. This is because the process</div><div class='line' id='LC208'>inherently involves a shower of particles called a ``jet&#39;&#39;. This</div><div class='line' id='LC209'>``jet clustering&#39;&#39; is a well-established technique employed at</div><div class='line' id='LC210'>many different particle physics experiments worldwide, and is</div><div class='line' id='LC211'>implemented in a common software framework called </div><div class='line' id='LC212'><span class="nb">{</span><span class="k">\tt</span> fastjet<span class="nb">}</span>~<span class="k">\cite</span><span class="nb">{</span>fastjet<span class="nb">_</span>manual<span class="nb">}</span>. </div><div class='line' id='LC213'>The mathematical problem is analogous to the</div><div class='line' id='LC214'>``K-nearest neighbors algorithm&#39;&#39;~<span class="k">\cite</span><span class="nb">{</span>knn<span class="nb">_</span>ieee<span class="nb">}</span> (kNN). </div><div class='line' id='LC215'>The single-core optimization of</div><div class='line' id='LC216'>jet clustering</div><div class='line' id='LC217'>is outlined in Ref.~<span class="k">\cite</span><span class="nb">{</span>fastjet<span class="nb">_</span>timing<span class="nb">}</span>. In a single core,</div><div class='line' id='LC218'>the computational time scales as <span class="s">$</span><span class="nb">O</span><span class="o">(</span><span class="nb">N^</span><span class="m">2</span><span class="o">)</span><span class="s">$</span> or  <span class="s">$</span><span class="nb">O</span><span class="o">(</span><span class="nb">N </span><span class="nv">\ln</span><span class="nb">{N}</span><span class="o">)</span><span class="s">$</span>, where <span class="s">$</span><span class="nb">N</span><span class="s">$</span></div><div class='line' id='LC219'>is the number of inputs to the algorithm, which scales linearly with</div><div class='line' id='LC220'>luminosity. </div><div class='line' id='LC221'><br/></div><div class='line' id='LC222'>There is existing work and literature on the topic of the</div><div class='line' id='LC223'>parallelization of the kNN algorithm, for instance, in</div><div class='line' id='LC224'>Refs.~<span class="k">\cite</span><span class="nb">{</span>knn<span class="nb">_</span>gpu<span class="nb">_</span>1, knn<span class="nb">_</span>gpu<span class="nb">_</span>2, knn<span class="nb">_</span>gpu<span class="nb">_</span>3<span class="nb">}</span>, where improvements</div><div class='line' id='LC225'><span class="s">$</span><span class="nb">O</span><span class="o">(</span><span class="m">100</span><span class="o">)</span><span class="s">$</span> in CPU performance are observed over standard</div><div class='line' id='LC226'>algorithms. Since the proposed use case is</div><div class='line' id='LC227'>very similar to the kNN algorithm, similar improvements to the</div><div class='line' id='LC228'>processing time by parallelization strategies are expected. </div><div class='line' id='LC229'><br/></div><div class='line' id='LC230'>We now discuss specific strategies that can be developed </div><div class='line' id='LC231'>to optimize performance in this algorithm. </div><div class='line' id='LC232'><br/></div><div class='line' id='LC233'><span class="k">\begin</span><span class="nb">{</span>figure<span class="nb">}</span>[h!]</div><div class='line' id='LC234'>&nbsp;&nbsp;&nbsp;&nbsp;<span class="k">\centering</span></div><div class='line' id='LC235'>&nbsp;&nbsp;&nbsp;&nbsp;<span class="k">\includegraphics</span><span class="na">[width=100mm]</span><span class="nb">{</span>lumitpeSingleMu-fitted2.png<span class="nb">}</span></div><div class='line' id='LC236'>&nbsp;&nbsp;&nbsp;&nbsp;<span class="k">\caption</span><span class="nb">{</span><span class="k">\label</span><span class="nb">{</span>lumitpeSingleMu<span class="nb">}</span> Event processing time versus</div><div class='line' id='LC237'>&nbsp;&nbsp;&nbsp;&nbsp;&nbsp;&nbsp;instantaneous luminosity.<span class="nb">}</span></div><div class='line' id='LC238'><span class="k">\end</span><span class="nb">{</span>figure<span class="nb">}</span></div><div class='line' id='LC239'><br/></div><div class='line' id='LC240'><br/></div><div class='line' id='LC241'><br/></div><div class='line' id='LC242'><span class="k">\subsection</span><span class="nb">{</span>Full Stack Parallelization<span class="nb">}</span></div><div class='line' id='LC243'><br/></div><div class='line' id='LC244'>To achieve the necessary improvements in performance required for scalability</div><div class='line' id='LC245'>of jet clustering, we propose to examine parallelization opportunities across</div><div class='line' id='LC246'>the entire software stack, including three specific areas : </div><div class='line' id='LC247'>(1) the use of lightweight concurrency</div><div class='line' id='LC248'>extraction to mask high-latency computations or I/O actions, </div><div class='line' id='LC249'>(2) extraction of</div><div class='line' id='LC250'>parallelization from the computation itself in the form of optimistic speculation</div><div class='line' id='LC251'>and specialized transform, and</div><div class='line' id='LC252'>(3) new methods for distributing the computation to</div><div class='line' id='LC253'>maximize parallelization on each node. </div><div class='line' id='LC254'><br/></div><div class='line' id='LC255'><br/></div><div class='line' id='LC256'><span class="k">\bigskip</span></div><div class='line' id='LC257'><span class="k">\noindent</span></div><div class='line' id='LC258'><span class="nb">{</span><span class="k">\bf</span>  Lightweight Concurrency for Latency Masking<span class="nb">}</span></div><div class='line' id='LC259'><span class="k">\bigskip</span></div><div class='line' id='LC260'><br/></div><div class='line' id='LC261'>Many mathematical kernels contain opportunities for extracting ``micro parallelism,&#39;&#39;</div><div class='line' id='LC262'>usually on the order of tens of instructions, from their computational components. </div><div class='line' id='LC263'>Unfortunately, it is very difficult to parallelize this computation profitably as</div><div class='line' id='LC264'>the overhead of thread creation, scheduling, synchronization, and migration outweigh</div><div class='line' id='LC265'>the gains in parallelism. Instead of extracting explicit parallelism from such</div><div class='line' id='LC266'>computations, we propose to explore methods of lightweight asynchrony to allow for</div><div class='line' id='LC267'>computation to proceed while waiting on high latency I/O operations to complete or</div><div class='line' id='LC268'>the results of other computations. Since the creation of threads and associated</div><div class='line' id='LC269'>schedule and synchronization costs are typically prohibitive, we will explore new</div><div class='line' id='LC270'>threading models that allow for logically-distinct computations to execute within</div><div class='line' id='LC271'>a given construct. The PIs previous research has indicated that such schemes can profitably</div><div class='line' id='LC272'>boost overall performance in the context of ML code~<span class="k">\cite</span><span class="nb">{</span>acml, parasites<span class="nb">}</span>. </div><div class='line' id='LC273'><span class="c">%The salient research</span></div><div class='line' id='LC274'><span class="c">%challenges in applying this strategy are as follows:  1) identifying what computation can be executed</span></div><div class='line' id='LC275'><span class="c">%safely during high latency operations at compile time, 2) providing a lightweight threading runtime</span></div><div class='line' id='LC276'><span class="c">%and programming model in the context of an imperative language, </span></div><div class='line' id='LC277'><span class="c">%3) specializing the approach to numeric kernels,</span></div><div class='line' id='LC278'><span class="c">%and 4) building support for computation in a distributed setting.</span></div><div class='line' id='LC279'><br/></div><div class='line' id='LC280'><span class="k">\bigskip</span></div><div class='line' id='LC281'><span class="k">\noindent</span></div><div class='line' id='LC282'><span class="nb">{</span><span class="k">\bf</span> Speculative Computation<span class="nb">}</span></div><div class='line' id='LC283'><span class="k">\bigskip</span></div><div class='line' id='LC284'><br/></div><div class='line' id='LC285'>In addition to exploring explicit parallelization of the numeric kernels in jet clustering, we</div><div class='line' id='LC286'>propose to explore extraction of parallelism via speculative computation. At its core, speculative</div><div class='line' id='LC287'>computation breaks apart sequential or parallel tasks into smaller tasks to be run in parallel. Once</div><div class='line' id='LC288'>the speculation has completed, the runtime system validates the computation. If the computation is</div><div class='line' id='LC289'>incorrect (<span class="nb">{</span><span class="k">\em</span> i.e.<span class="nb">}</span> a ``datarace&#39;&#39; is detected, the computation cannot be serialized, <span class="nb">{</span><span class="k">\em</span> etc.<span class="nb">}</span>), the</div><div class='line' id='LC290'>incorrect computation is re-executed in a non-speculative manner.  If the rate of mis-speculation</div><div class='line' id='LC291'>is low, such techniques can be leveraged to extract additional</div><div class='line' id='LC292'>parallelism. </div><div class='line' id='LC293'><span class="c">%There have been many different</span></div><div class='line' id='LC294'><span class="c">%proposals, including large efforts on transactional memory~\cite{},</span></div><div class='line' id='LC295'><span class="c">%lock elision~\cite{}, thread level speculation and speculative</span></div><div class='line' id='LC296'><span class="c">%multithreading~\cite{},</span></div><div class='line' id='LC297'><span class="c">% for integrating speculative computation into programming languages</span></div><div class='line' id='LC298'><span class="c">% and their associated runtimes~\cite{}.</span></div><div class='line' id='LC299'>The PIs have extensive experience with transactional memory~<span class="k">\cite</span><span class="nb">{</span>trans<span class="nb">}</span>, lightweight rollback methods~<span class="k">\cite</span><span class="nb">{</span>stab<span class="nb">}</span>, </div><div class='line' id='LC300'>leveraging memoization to reduce re-computation costs~<span class="k">\cite</span><span class="nb">{</span>memo1, memo2<span class="nb">}</span>,</div><div class='line' id='LC301'>and deterministic speculation~<span class="k">\cite</span><span class="nb">{</span>iso<span class="nb">}</span>. We propose to explore a specialized speculation framework leveraging different</div><div class='line' id='LC302'>speculation strategies, including speculation extracted by the programmer via programming language primitives,</div><div class='line' id='LC303'>library level speculation, and compiler extracted speculation.</div><div class='line' id='LC304'><span class="c">%The salient research</span></div><div class='line' id='LC305'><span class="c">%challenges in applying this strategy are as follows:  1) identification the appropriate speculation model and</span></div><div class='line' id='LC306'><span class="c">%discovering speculation points at compile time, 2) providing a speculative runtime specialized for jet clustering</span></div><div class='line' id='LC307'><span class="c">%and capable of realizing user, library, and compiler injected speculation, and 3) exploring new and specialized</span></div><div class='line' id='LC308'><span class="c">%lightweight validation and re-execution mechanisms, including validation across multiple speculation strategies.</span></div><div class='line' id='LC309'><br/></div><div class='line' id='LC310'><br/></div><div class='line' id='LC311'><br/></div><div class='line' id='LC312'><span class="k">\bigskip</span></div><div class='line' id='LC313'><span class="k">\noindent</span></div><div class='line' id='LC314'><span class="nb">{</span><span class="k">\bf</span> Smart Distribution<span class="nb">}</span></div><div class='line' id='LC315'><span class="k">\bigskip</span></div><div class='line' id='LC316'><br/></div><div class='line' id='LC317'>In order to increase parallelism, we will explore the use of the MapReduce</div><div class='line' id='LC318'>execution framework~<span class="k">\cite</span><span class="nb">{</span>mapreduce-osdi, mapreduce-hadoop<span class="nb">}</span>. MapReduce is a</div><div class='line' id='LC319'>runtime system recently developed for large-scale parallel data processing. It</div><div class='line' id='LC320'>enables programmers to easily deploy their applications on a cluster of</div><div class='line' id='LC321'>machines. Programmers only need to write two functions, Map and Reduce, and</div><div class='line' id='LC322'>submit these two functions as a job to the system. Then the MapReduce framework</div><div class='line' id='LC323'>takes care of all the aspects of the execution of the job. For example, the</div><div class='line' id='LC324'>framework packages and distributes the two functions over the cluster so that</div><div class='line' id='LC325'>the whole cluster can be utilized to execute the job; it also takes care of</div><div class='line' id='LC326'>fault-tolerance by monitoring the cluster during the execution of the job and</div><div class='line' id='LC327'>redistributes the job if some machine fails.</div><div class='line' id='LC328'><br/></div><div class='line' id='LC329'>Due to this simplicity and power, it is quickly gaining popularity in industry</div><div class='line' id='LC330'>for large-scale data processing. Many applications in scientific computing have not</div><div class='line' id='LC331'>yet explored the use of MapReduce in depth, however</div><div class='line' id='LC332'>previous research has explored implementing kNN with</div><div class='line' id='LC333'>MapReduce~<span class="k">\cite</span><span class="nb">{</span>knn-mapreduce-0, knn-mapreduce-1<span class="nb">}</span>. We intend to explore this</div><div class='line' id='LC334'>question in the context of jet clustering for the LHC.</div><div class='line' id='LC335'><br/></div><div class='line' id='LC336'><span class="k">\subsection</span><span class="nb">{</span>Summary<span class="nb">}</span></div><div class='line' id='LC337'><br/></div><div class='line' id='LC338'>In summary, the problem of expanding LHC computing to the exascale is</div><div class='line' id='LC339'>a difficult, but tractable one. This proposal investigates the</div><div class='line' id='LC340'>possibility of applying cutting-edge parallelization techniques such</div><div class='line' id='LC341'>as lightweight concurrency extraction, speculative computation, and</div><div class='line' id='LC342'>smarter distribution, to the</div><div class='line' id='LC343'>real-world application of LHC data processing.</div><div class='line' id='LC344'>The overall goal is to reduce the computational time for</div><div class='line' id='LC345'><span class="s">$</span><span class="nb">k</span><span class="s">$</span>-nearest-neighbor-like numerical kernels used for jet</div><div class='line' id='LC346'>clustering. The investigators of this</div><div class='line' id='LC347'>proposal have extensive experience in the various aspects of the</div><div class='line' id='LC348'>problem, and the synergistic application of this experience is</div><div class='line' id='LC349'>expected to attain considerable improvements in this area, which</div><div class='line' id='LC350'>are absolutely critical to the success of the future LHC physics</div><div class='line' id='LC351'>program. </div><div class='line' id='LC352'><br/></div><div class='line' id='LC353'><span class="k">\bibliographystyle</span><span class="nb">{</span>unsrt<span class="nb">}</span></div><div class='line' id='LC354'><span class="k">\bibliography</span><span class="nb">{</span>collaborative<span class="nb">_</span>proposal<span class="nb">}{}</span></div><div class='line' id='LC355'><span class="c">%\bibliography{auto_generated}</span></div><div class='line' id='LC356'><br/></div><div class='line' id='LC357'><span class="k">\end</span><span class="nb">{</span>document<span class="nb">}</span></div><div class='line' id='LC358'><br/></div><div class='line' id='LC359'><span class="c">% LocalWords:  Exascale LHC Ko Rappoccio Lukasz Zialek Ziarek EIC PIs</span></div><div class='line' id='LC360'><span class="c">% LocalWords:  Collider exascale transformative CMS multicore VMs bla</span></div><div class='line' id='LC361'><span class="c">% LocalWords:  luminosities HEP HTC HPC hadrons hadronic fastjet NN</span></div><div class='line' id='LC362'><span class="c">% LocalWords:  Ref kNN Refs knn gpu factorized workflow MapReduce mis</span></div><div class='line' id='LC363'><span class="c">% LocalWords:  datarace multithreading runtimes memoization</span></div></pre></div>
            </td>
          </tr>
        </table>
  </div>

  </div>
</div>

<a href="#jump-to-line" rel="facebox[.linejump]" data-hotkey="l" class="js-jump-to-line" style="display:none">Jump to Line</a>
<div id="jump-to-line" style="display:none">
  <form accept-charset="UTF-8" class="js-jump-to-line-form">
    <input class="linejump-input js-jump-to-line-field" type="text" placeholder="Jump to line&hellip;" autofocus>
    <button type="submit" class="button">Go</button>
  </form>
</div>

        </div>

      </div><!-- /.repo-container -->
      <div class="modal-backdrop"></div>
    </div><!-- /.container -->
  </div><!-- /.site -->


    </div><!-- /.wrapper -->

      <div class="container">
  <div class="site-footer">
    <ul class="site-footer-links right">
      <li><a href="https://status.github.com/">Status</a></li>
      <li><a href="http://developer.github.com">API</a></li>
      <li><a href="http://training.github.com">Training</a></li>
      <li><a href="http://shop.github.com">Shop</a></li>
      <li><a href="/blog">Blog</a></li>
      <li><a href="/about">About</a></li>

    </ul>

    <a href="/">
      <span class="mega-octicon octicon-mark-github"></span>
    </a>

    <ul class="site-footer-links">
      <li>&copy; 2013 <span title="0.06373s from github-fe120-cp1-prd.iad.github.net">GitHub</span>, Inc.</li>
        <li><a href="/site/terms">Terms</a></li>
        <li><a href="/site/privacy">Privacy</a></li>
        <li><a href="/security">Security</a></li>
        <li><a href="/contact">Contact</a></li>
    </ul>
  </div><!-- /.site-footer -->
</div><!-- /.container -->


    <div class="fullscreen-overlay js-fullscreen-overlay" id="fullscreen_overlay">
  <div class="fullscreen-container js-fullscreen-container">
    <div class="textarea-wrap">
      <textarea name="fullscreen-contents" id="fullscreen-contents" class="js-fullscreen-contents" placeholder="" data-suggester="fullscreen_suggester"></textarea>
          <div class="suggester-container">
              <div class="suggester fullscreen-suggester js-navigation-container" id="fullscreen_suggester"
                 data-url="/rappoccio/Exascale/suggestions/commit">
              </div>
          </div>
    </div>
  </div>
  <div class="fullscreen-sidebar">
    <a href="#" class="exit-fullscreen js-exit-fullscreen tooltipped leftwards" title="Exit Zen Mode">
      <span class="mega-octicon octicon-screen-normal"></span>
    </a>
    <a href="#" class="theme-switcher js-theme-switcher tooltipped leftwards"
      title="Switch themes">
      <span class="octicon octicon-color-mode"></span>
    </a>
  </div>
</div>



    <div id="ajax-error-message" class="flash flash-error">
      <span class="octicon octicon-alert"></span>
      <a href="#" class="octicon octicon-remove-close close ajax-error-dismiss"></a>
      Something went wrong with that request. Please try again.
    </div>

  </body>
</html>

